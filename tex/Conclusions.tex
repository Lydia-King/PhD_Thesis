\section{Conclusions and Future Work}
The aim of this thesis was to assess the role of genomic data (CNA data) in stratifying patients within predictive models for breast cancer OS and DSS outcomes. While studies of CNAs in breast cancer have been published in the literature and a large number of metrics defined to measure GI, limitations such as requiring access to raw or segmented data to calculate these measures, along with the complexity of some of these measures, have hindered their widespread use in research and clinical settings. This thesis has revealed the potential in incorporating CNA information, with lots of avenues for further research questions.  

In Chapter 2 we proposed a number of easy to interpret GI measures that can be calculated using publicly available summary CNA data. These CNA Score and Burden metrics captured the main aspects of CNAs, including magnitude, type and genomic location. Exploring the distributions of these CNA metrics, overall and stratified by PAM50 and IntClust molecular classifications, highlighted characteristic genomic aberrations documented previously, such as 5q deletions in Basal tumours and 17q amplification in HER2 tumours. In published research, subtypes associated with worse OS and DSS, i.e. Basal, HER2 and Luminal B subtypes, have higher GI than subtypes associated with better survival outcomes \citep{pmid22522925}. This result was echoed in the analysis here, when applying the new CNA metrics, but, in addition, we also reveal that subtypes associated with worse OS and DSS, tend to have significantly higher levels of deletions in genes than amplifications. 

Chapter 3 focused on how these CNA Score and Burden metrics are associated with survival, primarily DSS. Applying a combination of KM estimators, CPH regression and recursive partitioning survival trees, it was found that the global Absolute CNA Score metric was associated with DSS in Luminal A breast cancer patients. Patients with higher Absolute CNA Score values, i.e. Absolute CNA Score Quartile 4, indicative of higher levels of GI, had worse survival outcomes than patients with less GI, in Absolute CNA Score Quartiles 1-3 (Q1-3). This is encouraging, given reports in literature that the Luminal A and Luminal B subtypes may not be distinct subtypes, with ambiguity existing in DSS outcome for Luminal A patients \citep{pmid27341628, pmid26679376, pmid30849944, pmid37253056}, as it suggests that a simple measure of gene-centric CNAs across the genome can help identify Luminal A patients who are at elevated risk. These results are published in \textit{Survival Outcomes are Associated with Genomic Instability in Luminal Breast Cancer} \citep{King_2021}. We expanded the analysis further to assess the association between the 12 CNA Score and Burden metrics, formulated in this thesis, and DSS outcome, with consideration to PAM50 and IntClust molecular classifications. It was revealed that global CNA Del Score and Burden metrics, and chromosome arm CNA Del Score and Burden, specific to chromosomes 3p and 18q, play a role in stratifying patients on DSS outcome, primarily within Luminal A and Claudin-low patients. This again suggests that deletions are more harmful than amplifications and measuring these using CNA metrics can help identify patients with poorer survival outcomes. To facilitate navigation and exploration of the potential to unlock these discoveries, an R shiny app, GNOSIS, was developed to support the tractable and efficient exploration and application of survival analysis to cBioPortal clinical and genomic data products. This development is presented, accessible and published with peer review in \textit{GNOSIS: an R Shiny app supporting cancer genomics survival analysis with cBioPortal} \citep{King_GNOSIS}. 

While it was shown that our CNA Score and Burden metrics are accessible, easy to interpret and can provide valuable insights about how GI can impact survival outcomes, when comparing to pre-existing CNA measures, a number of limitations exist. These limitations, not encountered in most of the CNA measures discussed in Chapter 2, include that our CNA metrics are calculated using CNA summary data for annotated genes only, meaning the length of the CNA is not considered and CNAs occurring outside of gene regions are not included.  

To assess the effect that CNAs may have on gene expression, DGEA was carried out comparing the gene expression of groups of patients, stratified by similarities in survival outcome, where CNA metrics have a role to play in the stratification. This analysis, presented in Chapter 4, identified genes displaying significant differential gene expression between patient groups. This research revealed genes up-regulated among patients in survival tree nodes associated with poorer DSS outcomes include UBE2C, CXCL10 and S100P, while genes observed to be down-regulated among patients in survival tree nodes associated with poorer DSS outcomes include PIP, BCL2 and IRX2. Misexpression of these genes has been documented in literature as facilitating cell proliferation, tumour progression and invasion and as being correlated with survival \citep{pmid21195708, pmid31067633, pmid33681290, pmid20664598, pmid26560478, pmid30555735}. To investigate the direct relationship of a gene’s CNA state to the gene’s expression, this work employed a modified limma pipeline, comparing gene expression profiles across patients given the CNA state of the gene. A large number of genes were identified where the presence of a CNA in the gene led to an up- or down-regulation of that gene. As expected, if a deletion occurred the gene expression was usually down-regulated while if an amplification occurred the gene expression was usually up-regulated. Overall, using specified thresholds and considering sample size restrictions, 1,104 genes were differentially expressed in the three-gene specifications, ModLim3, and 3,197 genes were differentially expressed in the five-gene specification, ModLim5. Comparing these gene-sets to prognostic and predictive assays published in the literature indicated a moderate degree of congruence, identifying some of the same genes, but also identifying additional genes to be considered for further investigation as candidate biomarkers for breast cancer treatment and outcome.   

While we have shown that total copy number has a role in predictive models for survival outcome, there are a number of limitations associated with measuring copy number as a total across the two alleles, including masking of certain types of genomic aberrations and CNA changepoints. The aim of Chapters 5 and 6 was to generate allele-specific copy number data for the METABRIC patients and to detect regions of the genome where copy number changes of significant length occurred. To accomplish this a number of models, AIIM, AINIM, ADIM and ADNIM, were proposed and their performance assessed in a simulation study. Overall, based on a number of considerations, the ADIM model was selected and applied to the allele-specific copy number profiles. Unlike \cite{pmid32242091}, we retained valuable information on the type of CNA observed and the allele upon which the copy number change occurred, for each patient. Furthermore, rather than simply identify changepoints based on their observed frequency, \cite{GB}, our proposed models are based on $TS$ and $TE$ lengths for each changepoint category and are able to identify the changepoint categories accompanied by significantly large $TS$ or $TE$ lengths. Applying ADIM to the METABRIC cohort, Chapter 6, highlighted a number of genes and genomic regions where CNA changepoints, with large average $TS$ or $TE$ alterations, occurred. With application focusing on genomic lengths of genes, genes identified as containing changepoints with significant $TS$ and/or $TE$ lengths include OR52N1, TRIM5 and ALG1L2. Applying the ADIM with segmentation applied to the entire genomic region, genomic segments containing changepoints with significant $TS$ and/or $TE$ lengths include chromosome 1 segment 27, chromosome 1 segment 31 and chromosome 16 segment 9. Interestingly, comparing survival curves, for patients with/without the identified CNA changepoint of focus indicates that changepoints occurring in gene regions may be associated with poorer DSS, while changepoints occurring in genomic segments may be associated with improved DSS. These results with opposite direction of effect suggest that there is much still unknown and yet to explore with regards to CNAs, their changepoints and survival.

\subsection{Future Work}
While this thesis offers a comprehensive analysis of total and allele-specific copy number in the METABRIC cohort, and their associations with survival, the research offers opportunity for further work.  

Throughout this thesis we only consider OS and DSS outcomes, and not Recurrence-free survival (RFS). As RFS outcomes are also available for a large proportion of the METABRIC patients, a similar analysis assessing the association between our CNA Score and Burden metrics, and the allele-specific copy number changepoints, and RFS may lead to new insights. In addition, treatment information, including whether a patient received chemotherapy, radiotherapy, hormone therapy, and type of surgery, were not included in our analyses. Inclusion of this information may highlight CNA motifs that confer resistance to certain therapies. Although it should be noted that the METABRIC patients were enrolled between 1977 and 2005 and treatment options and standard of care have changed since e.g. use of Herceptin in HER2+ patients. Therefore, it may be interesting to produce and compare CNA Score and Burden metrics across patients in another more recent breast cancer dataset, for which detailed treatment information is available.

Further to this, assessing the performance of our predictive models including the CNA Score and Burden metrics, molecular subtype classifications and selected clinical information as candidate predictors, and comparison with performance of the predictive models only including the molecular subtype and clinical information, could be carried out. 

The gene expression analyses carried out using limma (the survival tree node analysis and CNA state analysis) were performed including only gene expression and CNA state in our models. In addition, no batch correction was carried out. Future work may include application of other gene expression methodologies, e.g. Significance Analysis of Microarrays, expansion to consider other variables of interest, implementation of batch correction techniques, and the possibility of combining gene expression data and our CNA metrics in predictive models for survival outcome.     

In application of the changepoint detection, the definition of a NoChangepoint region is a region of no alteration in CNA state in an allele, i.e. a constant CNA state observed within that region. This NoChangepoint region, of course, would be a constant state of neutral “normal” copy number, but would also represent a constant state of increased copy number, amplification, or a constant state of decreased copy number, deletion. A constant state of amplification or constant state deletion is still a region of copy number alteration, in contrast to normal copy number. While the application of the model to consecutive segmented regions as a form of search across the genome for changepoints does not necessarily require this as a point of focus, an expansion of the modelling approach could be considered, in order to include the three Nochangepoint categories, NoChangepoint, NoChangepoint\_Amp and NoChangepoint\_Del. In addition, during preprocessing of the ASCAT data the copy numbers of each allele were bound in the range [0-2], resulting in copy number changes for amplified regions being missed. These changepoints could potentially provide valuable insights and their impact on survival and treatment response should be investigated in future work. 

Application of the AD models, were under usual default modelling assumptions of \texttt{lm()} and default priors when fitting \texttt{MCMCglmm()}. Although the \texttt{MCMCglmm()} function enables fitting of random effects, applicability of random effects to the METABRIC data were not given further consideration within the scope of this body of research. Further work could give opportunity to explore suitability of assumptions and assumed priors, and application of random effects structures if required. 

The methods and results in this thesis, researching the role of CNA metrics and allele specific information in prognostic models, has led to a number of interesting outcomes and produced lists of candidate genes or genomic regions from which further exploration, in a research or clinical setting, can be carried out. One of the most important avenues of future work is the validation of these results in another breast cancer dataset, confirming our findings are dataset independent and of potential benefit. The code to run the analyses presented in this thesis is provided on GitHub at: https://github.com/Lydia-King/PhD\_Thesis.